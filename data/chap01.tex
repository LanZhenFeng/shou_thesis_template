% !TeX root = ../main.tex

\chapter{模版说明}

\section{格式标准}

该模版按照上海海洋大学研究生学位论文要求的格式制作。
参考文献的排版按照标准~\nospacecite{gbt7714-2005}执行。

\section{使用方法}

下面介绍该模版的使用方法.



\subsection{文件说明}

该模版包含文件结构如表~\ref{tab:file}所示。

引用时请使用自定义command \cs{nospacecite}\nospacecite{transt},而不是\cs{cite} \cite{transt}。
\begin{table}[htp]
    \centering

    \bicaption{文件说明}{Description of the document.} \label{label:figxx}
    \begin{tabular}{ll}
      \toprule
 \fixedrowheight     文件名          & 描述                         \\
      \midrule
 \fixedrowheight     shouthesis.cls   & 模板文件                     \\
 \fixedrowheight     shoulogo.log     & 学校标志的图片 \\
 \fixedrowheight     main.tex & 主文件 \\
\fixedrowheight      ref.bib & BibTeX文件    \\
  \fixedrowheight    data/ & 包含具体TeX文件的文件夹 \\
  \fixedrowheight    figures/ & 默认用来存放图片的文件夹 \\
  \fixedrowheight    data/chap01.tex & 第一章的TeX文件(建议每章一个TeX文件)\\
  \fixedrowheight    data/abstract.tex & 中英文摘要 \\
  \fixedrowheight    data/acknowledgements.tex & 致谢 \\
  \fixedrowheight    data/appendix.tex & 附录 \\
      \bottomrule
    \end{tabular}
    \label{tab:file}
  \end{table}
  
\subsection{编译命令}

推荐的编译命令为``\cmd{latexmk -xelatex -synctex=1 main.tex}''。
``\cmd{latexmk}''命令的运行需要系统安装有Perl解释器。
可以使用命令``\cmd{latexmk -C}''来删除编译产生的文件,可以使用命令``\cmd{latexmk -c}''来删除编译产生的临时文件。

\section{公式}
\begin{equation} \label{eq:1}
\begin{gathered}
  (\widehat{x_{tl}} ,\widehat{y_{tl} } )=(\sum_{y=0}^{H}\sum_{x=0}^{W}x\cdot   {P_{tl}}(x,y), \sum_{y=0}^{H}\sum_{x=0}^{W}y\cdot   {P_{tl}}(x,y)),  \\
  (\widehat{x_{br}} ,\widehat{y_{br} } )=(\sum_{y=0}^{H}\sum_{x=0}^{W}x\cdot   {P_{br}}(x,y), \sum_{y=0}^{H}\sum_{x=0}^{W}y {P_{br}}(x,y))\\
\end{gathered}
\end{equation}

\section{插图}
图片通常在 \env{figure} 环境中使用 \cs{includegraphics} 插入,如图~\ref{fig:example} 的源代码。
建议矢量图片使用 PDF 格式,比如数据可视化的绘图;
照片应使用 JPG 格式;
其他的栅格图应使用无损的 PNG 格式。
注意,LaTeX 不支持 TIFF 格式;EPS 格式已经过时。

\begin{figure}
    \centering
    \includegraphics[width=0.5\linewidth]{example-image-a.pdf}

    \bicaption{中文标题中文标题中文标题中文标题中文标题中文标题中文标题中文标题中文标题中文标题中文标题中文标题中文标题中文标题}{English titleEnglish titleEnglish titleEnglish titleEnglish titleEnglish titleEnglish titleEnglish titleEnglish titleEnglish titleEnglish title} \label{label:figxx}
    \label{fig:example}
\end{figure}
  