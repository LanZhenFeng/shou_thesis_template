% !TEX program = xelatex
% !TeX encoding = UTF-8
% !TeX spellcheck = en_US
% !BIB program = bibtex

\documentclass[oneside]{shouthesis}
% 如果不需要双面打印时插入的空白页,可以将twoside选项改成oneside
\renewcommand{\thefigure}{\thechapter-\arabic{figure}}
\renewcommand{\thetable}{\thechapter-\arabic{table}}
% ======= 根据自己需要加载相应宏包 ==========
\usepackage{mathrsfs}       % 提供mathscr: 花体
\usepackage{amsmath}        % AMSLaTeX宏包 用来排出更加漂亮的公式
\usepackage{amssymb}        % ams数学符号
\usepackage{physics}        % 物理常用符号
\usepackage{threeparttable} % 表格加脚注
\usepackage{booktabs}       % 表格,横的粗线;\specialrule{1pt}{0pt}{0pt}
\usepackage{longtable}      %支持跨页的表格。
\usepackage[justification=centering]{caption}
% =================================
\setcitestyle{super} %角标格式右上
% =================================
% 定义所有的图片文件在 figures 子目录下
% 如果图片放在主目录,则可以注释下面的命令
\graphicspath{{figures/}}
% =================================

% =================================
% 论文封面需要的基本信息配置
% 注意:
%   1. '\shousetup'配置里面不要出现空行 (很重要,不然会编译出错!)
%   2. 后面带*的是相应的英文参数。比如author是中文名,author*是英文名
% =================================
\shousetup{
    % 中文标题在封面上分为两行显示,分别为titleA和titleB。如果一行放得下,titleB留空白即可。
    titleA  = {上海海洋大学学硕士学位论文},
    titleB  = {\LaTeX{} 模板使用示例文档 v\version},
    title*  = {An Introduction to \LaTeX{} Thesis Template of Shanghai Ocean
              University v\version},
    % 专业
    discipline  = {计算机},
    discipline* = {Physics},
    % 姓名
    author  = {张三},
    author* = {San Zhang},
    % 指导教师,中文姓名和职称之间空格分开,下同
    supervisor  = {李四\ 教授},
    supervisor* = {Professor Si Li},
    % 日期,如果不指定则默认显示编译当月
    % date = {2022年4月},
    % data* = {April, 2022}
}
% =================================

\begin{document}

\maketitle

\frontmatter
% 摘要
% !TeX root = ../thuthesis-example.tex

% 中英文摘要和关键字

\begin{abstract}
 {\zihao{-4} 单目标跟踪是计算机视觉领域的基础任务之一,在视频安防、自动驾驶和无人机跟踪等领域具有广泛应用。

}
  
    \shousetup{
        keywords = {目标跟踪、 轻量化跟踪、 提示学习、 水下目标跟踪、 水下跟踪数据集},
    }
\end{abstract}

\begin{abstract*}
{\zihao{-4}	Single object tracking is a fundamental task in computer vision, with widespread applications in video surveillance, autonomous driving, and drone tracking.

}
  
  % Use comma as separator when inputting
    \shousetup{
        keywords* = {object tracking, efficient object tracking, prompt learning, underwater object tracking, underwater object tracking datasets},
    }
\end{abstract*}

\newpage
\mbox{}

\newpage
% 中文目录
\tableofcontents    
\thispagestyle{empty} % 不记录页码的页面

% 正文部分
\mainmatter

% 包含章节

% !TeX root = ../main.tex

\chapter[引言]{引~~~~~~言}
\vspace{1em}

这是引言



    % 插入 intro.tex 文件内容
% !TeX root = ../main.tex

\chapter{模版说明}

\section{格式标准}

该模版按照上海海洋大学研究生学位论文要求的格式制作。
参考文献的排版按照标准~\nospacecite{gbt7714-2005}执行。

\section{使用方法}

下面介绍该模版的使用方法.



\subsection{文件说明}

该模版包含文件结构如表~\ref{tab:file}所示。

引用时请使用自定义command \cs{nospacecite}\nospacecite{transt},而不是\cs{cite} \cite{transt}。
\begin{table}[htp]
    \centering

    \bicaption{文件说明}{Description of the document.} \label{label:figxx}
    \begin{tabular}{ll}
      \toprule
 \fixedrowheight     文件名          & 描述                         \\
      \midrule
 \fixedrowheight     shouthesis.cls   & 模板文件                     \\
 \fixedrowheight     shoulogo.log     & 学校标志的图片 \\
 \fixedrowheight     main.tex & 主文件 \\
\fixedrowheight      ref.bib & BibTeX文件    \\
  \fixedrowheight    data/ & 包含具体TeX文件的文件夹 \\
  \fixedrowheight    figures/ & 默认用来存放图片的文件夹 \\
  \fixedrowheight    data/chap01.tex & 第一章的TeX文件(建议每章一个TeX文件)\\
  \fixedrowheight    data/abstract.tex & 中英文摘要 \\
  \fixedrowheight    data/acknowledgements.tex & 致谢 \\
  \fixedrowheight    data/appendix.tex & 附录 \\
      \bottomrule
    \end{tabular}
    \label{tab:file}
  \end{table}
  
\subsection{编译命令}

推荐的编译命令为``\cmd{latexmk -xelatex -synctex=1 main.tex}''。
``\cmd{latexmk}''命令的运行需要系统安装有Perl解释器。
可以使用命令``\cmd{latexmk -C}''来删除编译产生的文件,可以使用命令``\cmd{latexmk -c}''来删除编译产生的临时文件。

\section{公式}
\begin{equation} \label{eq:1}
\begin{gathered}
  (\widehat{x_{tl}} ,\widehat{y_{tl} } )=(\sum_{y=0}^{H}\sum_{x=0}^{W}x\cdot   {P_{tl}}(x,y), \sum_{y=0}^{H}\sum_{x=0}^{W}y\cdot   {P_{tl}}(x,y)),  \\
  (\widehat{x_{br}} ,\widehat{y_{br} } )=(\sum_{y=0}^{H}\sum_{x=0}^{W}x\cdot   {P_{br}}(x,y), \sum_{y=0}^{H}\sum_{x=0}^{W}y {P_{br}}(x,y))\\
\end{gathered}
\end{equation}

\section{插图}
图片通常在 \env{figure} 环境中使用 \cs{includegraphics} 插入,如图~\ref{fig:example} 的源代码。
建议矢量图片使用 PDF 格式,比如数据可视化的绘图;
照片应使用 JPG 格式;
其他的栅格图应使用无损的 PNG 格式。
注意,LaTeX 不支持 TIFF 格式;EPS 格式已经过时。

\begin{figure}
    \centering
    \includegraphics[width=0.5\linewidth]{example-image-a.pdf}

    \bicaption{中文标题中文标题中文标题中文标题中文标题中文标题中文标题中文标题中文标题中文标题中文标题中文标题中文标题中文标题}{English titleEnglish titleEnglish titleEnglish titleEnglish titleEnglish titleEnglish titleEnglish titleEnglish titleEnglish titleEnglish title} \label{label:figxx}
    \label{fig:example}
\end{figure}
  

% !TeX root = ../main.tex




\chapter*{小~~~~~~结}
\addcontentsline{toc}{chapter}{小结}

\vspace{1em}

这是小结  
\clearpage

\phantomsection
\addcontentsline{toc}{chapter}{参考文献}
\bibliographystyle{gbt7714-2005-numerical}
\bibliography{ref}  % 参考文献BibTeX文件名称,默认为ref.bib

% 附录,所发文章与参与项目
% !TeX root = ../main.tex

\chapter*{附~~~~~~录}
\addcontentsline{toc}{chapter}{附录}

\begin{appendix}
\vspace{1em}




写作规范:

附录包括放在正文内显得过分冗长的公式推导;供查读方便所需的辅助性数学工具或表格;重复性数据图表;论文使用的主要符号、意义、单位、缩写、程序全文及说明等。 

“附录”二字居中打印(三号黑体,两字间空两格),空一行打印附录内容。

\end{appendix}

发表的学术论文:
\begin{enumerate}
    \item 论文信息。
\end{enumerate}

  
% 致谢
% !TeX root = ../main.tex

\begin{acknowledgements}

\vspace{1em}

本研究获得国家 863 计划(2012AA10A400)、国家自然科学基金(31472278)上海大学生知识服务平台(ZF1206)的资助。

……

写作规范:

致谢主要是对给予各类资助、指导和协助完成研究工作,以及提供各种对论文工作有利条件的单位及个人表示感谢。致谢应实事求是,切忌浮夸与庸俗之词。

“致谢”二字居中打印(三号黑体,两字间空两格),换行打印致谢内容。


\end{acknowledgements}
 

\end{document}
